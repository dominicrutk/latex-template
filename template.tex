\documentclass{article}

\usepackage[utf8]{inputenc}
\usepackage{algorithm}
\usepackage{algpseudocode}
\usepackage{amsmath}
\usepackage{amssymb}
%\usepackage{amsthm}
%\usepackage{babel}
%\usepackage[
%    backend=biber,
%    style=alphabetic,
%    sorting=nty
%]{biblatex}
\usepackage{bm}
\usepackage{cancel}
\usepackage{comment}
\usepackage{diagbox}
\usepackage[shortlabels]{enumitem}
\usepackage[margin=1in]{geometry}
\usepackage{hyperref}
\usepackage{listings}
\usepackage{mathtools}
%\usepackage{qtree}
%\usepackage{tikz}
%\usetikzlibrary{positioning, shapes.misc}

%\addbibresource{bibliography.bib}

\newcommand{\newp}{\medskip\noindent}
\newcommand{\Mod}[1]{\ \mathrm{mod}\ #1}
\newcommand{\PMod}[1]{\ (\mathrm{mod}\ #1)}
\newcommand{\N}{\mathbb{N}}
\newcommand{\Z}{\mathbb{Z}}
\newcommand{\Q}{\mathbb{Q}}
\newcommand{\R}{\mathbb{R}}
\newcommand{\C}{\mathbb{C}}
\DeclareMathOperator{\lcm}{lcm}
%\renewcommand{\Pr}[1]{\text{Pr}[#1]}
%\newcommand{\E}[1]{\mathbb{E}[#1]}
%\newcommand{\Var}[1]{\text{Var}(#1)}
%\newcommand{\PrB}[1]{\text{Pr}\left[#1\right]}
%\newcommand{\EB}[1]{\mathbb{E}\left[#1\right]}
%\newcommand{\VarB}[1]{\text{Var}\left(#1\right)}
%\newcommand{\BR}[1]{\bm{\mathrm{#1}}}

\title{<%= className %> <%= assignment %>}
<%_ if (listCollaborators) { _%>
\author{<%= authorName %> (<%= collaboratorStr %>)}
<%_ } else { _%>
\author{<%= authorName %>}
<%_ } _%>
\date{Due <%= dueDate %>}

\begin{document}

<% if (titlePage) { -%>
\begin{titlepage}
    \null
    \nointerlineskip
    \vfill
    \let \snewpage \newpage
    \let \newpage \relax
    \maketitle
    \let \newpage \snewpage
    \vfill
    \null\null\null\null\null\null\null
    \thispagestyle{empty}
\end{titlepage}
\pagenumbering{arabic}\setcounter{page}{2}
<% } else { -%>
\maketitle
<% } -%>

<% problemNames.slice(0, -1).forEach((problemName, i) => { -%>
%%%%%%%%%%%%%%%%%%%%%%%%%%%%%%%%%%%%%%%%%%%%%%%%%% PROBLEM <%= i + 1 %> %%%%%%%%%%%%%%%%%%%%%%%%%%%%%%%%%%%%%%%%%%%%%%%%%%
\section<% if (!namedProblems) { -%>*<% } -%>{<%= problemName %>}


%============================% SOLUTION <%= i + 1 %> %============================%
\subsection*{Solution}


\newpage
<% }); %>%%%%%%%%%%%%%%%%%%%%%%%%%%%%%%%%%%%%%%%%%%%%%%%%%% PROBLEM <%= problemNames.length %> %%%%%%%%%%%%%%%%%%%%%%%%%%%%%%%%%%%%%%%%%%%%%%%%%%
\section<% if (!namedProblems) { -%>*<% } -%>{<%= problemNames[problemNames.length - 1] %>}


%============================% SOLUTION <%= problemNames.length %> %============================%
\subsection*{Solution}


\end{document}
